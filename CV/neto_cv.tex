\documentclass[letterpaper]{article}

\usepackage{hyperref}
\usepackage{geometry}

% Comment the following lines to use the default Computer Modern font
% instead of the Palatino font provided by the mathpazo package.
% Remove the 'osf' bit if you don't like the old style figures.
% \usepackage[T1]{fontenc}
\usepackage[latin1]{inputenc}
% \usepackage[sc,osf]{mathpazo}
\usepackage[sc,osf]{mathpazo}

\def\name{Jos\'e F. Silva Neto}

% Replace this with a link to your CV if you like, or set it empty
% (as in \def\footerlink{}) to remove the link in the footer:
\def\footerlink{}

% The following metadata will show up in the PDF properties
\hypersetup{
  colorlinks = true,
  urlcolor = black,
  pdfauthor = {\name},
  pdfkeywords = {smt, automated deduction, logic},
  pdftitle = {\name: Curriculum Vitae},
  pdfsubject = {Curriculum Vitae},
  pdfpagemode = UseNone
}

\geometry{
  body={6.5in, 8.5in},
  left=1.0in,
  top=1.25in
}

% Customize page headers
\pagestyle{myheadings}
\markright{\name}
\thispagestyle{empty}

% Custom section fonts
\usepackage{sectsty}
\sectionfont{\rmfamily\mdseries\Large}
\subsectionfont{\rmfamily\mdseries\itshape\large}

% Other possible font commands include:
% \ttfamily for teletype,
% \sffamily for sans serif,
% \bfseries for bold,
% \scshape for small caps,
% \normalsize, \large, \Large, \LARGE sizes.

% Don't indent paragraphs.
\setlength\parindent{0em}

% Make lists without bullets
\renewenvironment{itemize}{
  \begin{list}{}{
      \setlength{\leftmargin}{1.5em}
    }
  }{
  \end{list}
}

\begin{document}

% Place name at left
{\huge \name}

% Alternatively, print name centered and bold:
% \centerline{\huge \bf \name}

\vspace{0.25in}

\begin{minipage}{0.75\linewidth}
  TASC Building 8004 \\
  School Of Computing Science\\
  Simon Fraser University 8888 University Drive Burnaby, B.C. V5A 1S6\\
  Canada
\end{minipage}
\begin{minipage}{0.45\linewidth}
  \begin{tabular}{ll}
    Phone: & +55 (84) 98113-3108 \\
    E-mail: & \href{mailto:netolcc06@gmail.com}{\tt jfdasilv@sfu.ca} \\
    Website: & https://netolcc06.github.io
  \end{tabular}
\end{minipage}

%\section*{Personal}

%\begin{itemize}
%  \item Birthday: 17/08/1988.
%  \item Nationality: Brazilian.
%  % \item[$\bullet$] \ Civil state: Single.
%\end{itemize}

\section*{Education and Qualifications}

\begin{itemize}
 
 \item \textbf{Simon Fraser University}, Burnaby, BC \\
 \textit{Ph.D.}, Computing Science, Fall 2018 - Present
 
 \item \textbf{Federal University of Rio Grande do Norte}, Natal, RN, Brazil \\
 \textit{Thesis}: Fuzzy Segmentation of Three-Dimensional Objects with Textural Properties \\
 \textit{B.SC, M.Sc,} Computer Science, 2009-2014
 
 %\item[$\bullet$] \textbf{M.S.}: Computer Science, focusing on Image Processing and Graphics. August 2012 to October 2014 \\
 %   \textbf{Institution}: Federal University of Rio Grande do Norte - UFRN\\
 %   \textbf{Advisor}: Bruno Motta de Carvalho.\\
  %  \textbf{Dissertation}:  Fuzzy Segmentation of Three-Dimensional Objects with Textural Properties.
  
%  \item[$\bullet$] \textbf{B.S.}: Computer Science. February 2009 to August 2012 \\
%  \textbf{Institution}: Federal University of Rio Grande do Norte - UFRN \\
%  \textbf{Capstone project}: Texture Fuzzy Segmentation using Adaptive Affinity Functions and Skew Divergence.


  %\item[$\bullet$] 1st place among 229 students that applied to the Computer Science course at UFRN in the year of 2009
  
\end{itemize}

\section*{Publications}

\begin{itemize}

  \item \textit{Vitor Godeiro; \textbf{Jos\'e F. S. Neto}; Bruno Motta De Carvalho; Julianny Barreto Ferraz; Bruno Santana Da Silva; Renata Antonaci Gama}. Chronic Wound Tissue Classification using Convolutional Networks and Color Space Reduction. To appear at Machine Learning for Signal Processing (MLSP) 2018.

  \item \textit{\textbf{Jos\'e F. S. Neto}, Waldson P. N. Leandro, Matheus A. Gadelha, Tiago S. Santos, Bruno M. Carvalho, Edgar Gardu\~no}. Texture Fuzzy Segmentation using Skew Divergence Adaptive Affinity Functions \textbf{(under review)}.

  \item \textit{Bruno M. Carvalho ; Edgar Gardu\~no ; Tiago S. Santos ; Lucas M. Oliveira ; \textbf{Jos\'e F. S. Neto} . Fuzzy segmentation of video shots using hybrid color spaces and motion
  information}. Pattern Analysis and Applications (Print), v. 17, p. 013-0359-1, 2013. \\

\end{itemize}

\section*{Research Experience}

\begin{itemize}

  \item \textbf{Chronic Wound Tissue Classification using Convolutional Networks and Color Space Reduction. (2018)}
  We investigated algorithms to perform the segmentation of wounds as well as the use of several convolutional networks for classifying tissue as Necrotic, Granulation or Slough. We extended four architectures: U-Net, Segnet, FCN8 and FCN32, and proposed a color space reduction methodology that increased the reported accuracies, specificities, sensitivities and Dice coefficients for all 4 networks, achieving very good levels.
  UFRN. Advisor: Bruno Motta de Carvalho.

  \item \textbf{Automatic texture segmentation. (2017)}
  Developed a method to automatically  choose  the seeds for our semi-automatic texture segmentation algorithm.
  UFRN. Advisor: Bruno Motta de Carvalho.

  \item \textbf{Fuzzy Segmentation of Three-Dimensional Objects with Textural Properties. (2012-2014)}
  Extended the MOFS algorithm in order to achieve three-dimensional texture segmentation. Performed experiments with synthetic and real data obtained from the Multimodal Brain Tumor Segmentation dataset from the Medical Image Computing and Computer Assisted Intervention (MICCAI) Conference 2012.
  UFRN. Advisor: Bruno Motta de Carvalho.

  \item \textbf{Texture Fuzzy Segmentation using Adaptive Affinity Functions and Skew Divergence. (2011-2012)}
  Capstone project of my bachelor in Computer Science. This work discusses how affinity functions
  can be used as texture descriptors, presenting a fuzzy segmentation algorithm that employs the Skew
  Divergence and the Gaussian Distribution as affinity functions, comparing the results obtained using
  these approaches.
  UFRN. Advisor: Bruno Motta de Carvalho.
  
  \item \textbf{National Laboratory for Scientific Computing, Petropolis-RJ. July 2009}
  Development of a Multithread Library System (C++ and Windows) for a Remote Rendering Project. 
  Advisor: Selan Rodrigues dos Santos. Supervisor: Jauvane Cavalcante de Oliveira.
  
\end{itemize}

\section*{Teaching}

\begin{itemize}

  \item \textit{Algorithms and Data Structures. (2009 - 2010)}
  - Taught weekly discussion sessions for 2 classes (40 students in total)
  - Instructed students with C++ projects
  UFRN. Supervisor: Selan Rodrigues dos Santos.
  
  \item \textit{Elements of Mathematics for Computer Science. Fall 2013}
  - Taught Combinatorial Analysis and Probability for 1 class (30 students in total)
  - Elaborated materials and classes about these topics.
  UFRN. Supervisors: Bruno Motta de Carvalho and Jo\~ao Marcos de Almeida. 
  
  \item \textit{Game Development with XNA - Summer School. UFRN, 2011}
  - Course of 2 weeks for a class with 40 students
  - 2D Side Scroller development
  - 3D Fundamentals(Camera Development, Illumination and HLSL)
  
  \item \textit{Tutoring Education Program(PET)}
  
  In this program we executed teaching activities (such as minicourses, lectures, teaching assistance), research (undergraduate research as volunteers) and extension activities outside the University.
 
\end{itemize}

\section*{Computer Skills}

%\begin{itemize}
 \textbf{Main Skills:} Computer Graphics, Image Processing, Computer Vision.\\
 \textbf{Languages:} C/C++, Python, C\#, Java, Lua, Matlab.\\
 \textbf{Libraries:} : OpenGL, OpenCV, Numpy.\\

%\end{itemize}

\bigskip
% Footer
\begin{center}
  \begin{footnotesize}
    Last update: \today \\
    \href{\footerlink}{\texttt{\footerlink}}
  \end{footnotesize}
\end{center}

\end{document}
